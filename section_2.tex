%%%%%%%%%%%%%%%%%%%%%%%%%%%%%%%%%%%%%%%%%%%%%%%%%%%%%%%%%%%%%%%%%%%%%%%%%%%%%%%%%%%%%%
%%%%%%%%%%%%%%%%%%%%%%%%%%%%%%% Obrisati ovaj dio %%%%%%%%%%%%%%%%%%%%%%%%%%%%%%%%%%%%
%%%%%%%%%%%%%%%%%%%%%%%%%%%%%%%%%%%%%%%%%%%%%%%%%%%%%%%%%%%%%%%%%%%%%%%%%%%%%%%%%%%%%%

\section{Primjeri}

\subsection{Slika}
\begin{figure}[h]
	\begin{center}
		\includegraphics[width=0.3\textwidth]{pmf.png}
		\caption{Logo PMF koji nije Word 2003 clip art}
	\end{center}
\end{figure}

\subsection{Zadatak}
% U preamble.tex je definisan \newtheorem{zadatak}{Zadatak}.
% Okruženje je automatski numerisano na isti način kao i \section itd.
% Komanda \dd daje pravilan znak diferencijala.
\begin{zadatak}
	\label{prvi-integral}
	Izračunati integral $I$ ako je $D$ oblast ograničena sa $y_1=2x^2, \ y_2 = 1+x^2$.
	\[
		I = \iint_D (x+2y) \ \dd{x} \dd{y}
		.\]
\end{zadatak}

% Tikz može da generiše grafike na osnovu jednačina.
\begin{figure}[H]
	\centering
	\begin{tikzpicture}
		\begin{axis}[	ymin=-1,ymax=3,xmax=2,xmin=-2,%xticklabel=\empty,yticklabel=\empty,
				minor tick num=1,axis on top,axis lines = middle,
				xlabel=$x$,ylabel=$y$,label style={at={(ticklabel cs:1.1)}}]
			\addplot[blue, samples=50, domain=-1.5:1.5]{2*x^2};
			\addplot[red, samples=50, domain=-1.5:1.5]{1+x^2};
			\addplot[name path=A, blue, samples=50, domain=-1:1]{2*x^2};
			\addplot[name path=B,red, samples=50, domain=-1:1]{1+x^2};
			\node[label={180:{(-1,2)}},circle,fill,inner sep=1pt] at (axis cs:-1,2) {};
			\node[label={180:{(1,2)}},circle,fill,inner sep=1pt] at (axis cs:1,2) {};
			\addplot[blue!15] fill between[of=A and B];
		\end{axis}
	\end{tikzpicture}
	\caption{Slika uz zadatak \ref{prvi-integral}}
\end{figure}

Prvi korak je odrediti granice: $2x^2=1+x^2 \implies x = \pm 1$. Tačke presjeka su dakle $(1,2)$ i $(-1,2)$.

\[
	x\Big|_{-1}^{1}, \qquad y \Big|_{2x^2}^{1+x^2}
	.\]

% Okruženje alignat vertikalno poravna znakove obilježene sa &.
% Bez zvjezdice daje numerisane linije.
% Numerisati samo jednu liniju se može sa \tag{1} na kraju linije.
\begin{alignat*}{1}
	I & = \int_{-1}^{1} \dd{x} \int_{2x^2}^{1+x^2}(x+2y) \dd{y}
	= \int_{-1}^{1} \dd{x} \left( xy\Big|_{2x^2}^{1+x^2} + 2y^2 \Big|_{2x^2}^{1+x^2} \right)          \\
	  & = \int_{-1}^{1} \left\{ x(1+x^2-2x^2) + 2 \left[ (1+x^2)^2 - (2x^2)^2 \right] \right\} \dd{x} \\
	  & = \ldots = \frac{32}{15} \tag{1}
	.\end{alignat*}


\subsection{Tabela}
% Preuzeto sa:
% https://latex-tutorial.com/tutorials/tables/
\begin{table}[h]
	\begin{center}
		\begin{tabular}{l|S|r}
			\toprule % <-- Toprule here
			\textbf{Value 1} & \textbf{Value 2} & \textbf{Value 3} \\
			$\alpha$         & $\beta$          & $\gamma$         \\
			\midrule % <-- Midrule here
			1                & 1110.1           & a                \\
			2                & 10.1             & b                \\
			3                & 23.113231        & c                \\
			\bottomrule % <-- Bottomrule here
		\end{tabular}
		\caption{Tabela s paketom booktabs.}
	\end{center}
\end{table}

\subsection{Kompleksnija tabela}
% Primjer tabele preuzet sa
% https://tex.stackexchange.com/questions/112343/beautiful-table-samples
\begin{table}[h]
	\centering
	\begin{tabular}{SSSSSSSS} \toprule
		{$m$} & {$\Re\{\underline{\mathfrak{X}}(m)\}$} & {$-\Im\{\underline{\mathfrak{X}}(m)\}$} & {$\mathfrak{X}(m)$} & {$\frac{\mathfrak{X}(m)}{23}$} & {$A_m$} & {$\varphi(m)\ /\ ^{\circ}$} & {$\varphi_m\ /\ ^{\circ}$} \\ \midrule
		1     & 16.128                                 & +8.872                                  & 16.128              & 1.402                          & 1.373   & -146.6                      & -137.6                     \\
		2     & 3.442                                  & -2.509                                  & 3.442               & 0.299                          & 0.343   & 133.2                       & 152.4                      \\
		3     & 1.826                                  & -0.363                                  & 1.826               & 0.159                          & 0.119   & 168.5                       & -161.1                     \\
		4     & 0.993                                  & -0.429                                  & 0.993               & 0.086                          & 0.08    & 25.6                        & 90                         \\ \midrule
		5     & 1.29                                   & +0.099                                  & 1.29                & 0.112                          & 0.097   & -175.6                      & -114.7                     \\
		6     & 0.483                                  & -0.183                                  & 0.483               & 0.042                          & 0.063   & 22.3                        & 122.5                      \\
		7     & 0.766                                  & -0.475                                  & 0.766               & 0.067                          & 0.039   & 141.6                       & -122                       \\
		8     & 0.624                                  & +0.365                                  & 0.624               & 0.054                          & 0.04    & -35.7                       & 90                         \\ \midrule
		9     & 0.641                                  & -0.466                                  & 0.641               & 0.056                          & 0.045   & 133.3                       & -106.3                     \\
		10    & 0.45                                   & +0.421                                  & 0.45                & 0.039                          & 0.034   & -69.4                       & 110.9                      \\
		11    & 0.598                                  & -0.597                                  & 0.598               & 0.052                          & 0.025   & 92.3                        & -109.3                     \\ \bottomrule
	\end{tabular}
	\caption{Kompleksna tabela}
\end{table}

\subsection{Mjerne jedinice}

Paket \texttt{siunits} dozvoljava da se lako i pravilno pišu veličine koje imaju mjerne jedinice. Kako se često pogrešno radi: $g = 9.81 ms^{-2}$.
Kako je pravilno: $g = \SI{9.81}{\meter\per\second\squared}$.

\subsection{Matematika}

\begin{zadatak}
	\label{12:zd-krint-kontura}
	Izračunati $\displaystyle \oint y^2 \dd{x} + 3xy \dd{y}$ na konturi sa slike \ref{12:slika-krint-kontura}.
\end{zadatak}

\begin{center}
	\begin{minipage}[t]{0.48\textwidth}
		\begin{figure}[H]
			\centering
			\begin{tikzpicture}
				\begin{axis}[ymin=-0.5,ymax=2.5,xmax=2.5,xmin=-2.5,%xticklabel=\empty,yticklabel=\empty,
						minor tick num=1,
						axis lines = middle,
						trig format plots=rad,
						unit vector ratio={1 1},
						xlabel=$x$,ylabel=$y$,label style={at={(ticklabel cs:1.1)}}]
					\addplot[name path=A,blue, samples=50, domain=0:pi]({2*cos(x)},{2*sin(x)});
					\addplot[name path=B,blue, samples=50, domain=0:pi]({cos(x)},{sin(x)});
					\addplot +[blue, mark=none] coordinates {(-2, 0) (-1, 0)};
					\addplot +[blue, mark=none] coordinates {(1, 0) (2, 0)};
				\end{axis}
			\end{tikzpicture}
			\caption{Slika uz zadatak \ref{12:zd-krint-kontura}}
			\label{12:slika-krint-kontura}
		\end{figure}
	\end{minipage}
	\begin{minipage}[t]{0.48\textwidth}
		\[
			\oint_C P\dd{x} + Q\dd{y} = \iint \left( \pdv{Q}{x} - \pdv{P}{y}  \right) \dd{x} \dd{y}
		\]
		\begin{alignat*}{3}
			Q & = 3xy, \qquad &  & \pdv{Q}{{x}} &  & = 3y \\
			P & = y^2, \qquad &  & \pdv{P}{{y}} &  & = 2y
		\end{alignat*}
		\[
			r\Big|_1^{2}, \qquad \vp\Big|_0^{\pi}
		\]
	\end{minipage}
\end{center}

\begin{alignat*}{1}
	\oint_C & = \iint_C (3y-2y) \dd{x} \dd{y} = \iint_C y \dd{x} \dd{y} \\
	        & = \int_{1}^{2} \int_{0}^{\pi} r^2 \sin\vp \dd{r} \dd{\vp}
	= \frac{r^3}{3} \bigg|_1^{2} (- \cos \vp) \bigg|_0^{\pi} = \frac{14}{3}
	.\end{alignat*}

\begin{zadatak}
	Ako je vektorsko polje oblika $\vb{F} = P \vb{i} + Q \vb{j} + R \vb{k}$ dokazati da je $\ddiv \rrot \vb{F} = 0$.
\end{zadatak}

\begin{alignat*}{1}
	\rrot \vb{F}       & =	\left( \pdv{R}{y} - \pdv{Q}{z}  \right) \vb{i} -
	\left( \pdv{R}{x} - \pdv{P}{z}  \right) \vb{j} +
	\left( \pdv{Q}{x} - \pdv{P}{y}  \right) \vb{k}                            \\
	\ddiv \rrot \vb{F} & =	\pdv{}{x} \left( \pdv{R}{y} - \pdv{Q}{z} \right) -
	\pdv{}{y} \left( \pdv{R}{x} - \pdv{P}{z} \right) +
	\pdv{}{x} \left( \pdv{Q}{x} - \pdv{P}{y} \right) = 0
	.\end{alignat*}