%%%%%%%%%%%%%%%%%%%%%%%%%%%%%%%%%%%%%%%%%%%%%%%%%%%%%%%%%%%%%%%%%%%%%%%%%%%%%%%%%%%%%%
%%%%%%%%%%%%%%%%%%%%%%%%%%%%%%% Obrisati ovaj dio %%%%%%%%%%%%%%%%%%%%%%%%%%%%%%%%%%%%
%%%%%%%%%%%%%%%%%%%%%%%%%%%%%%%%%%%%%%%%%%%%%%%%%%%%%%%%%%%%%%%%%%%%%%%%%%%%%%%%%%%%%%

\section{Primjeri}

\subsection{Slika}
\begin{figure}[h]
	\begin{center}
		\includegraphics[width=0.3\textwidth]{pmf.png}
		\caption{Logo PMF koji nije Word 2003 clip art}
	\end{center}
\end{figure}

\subsection{Zadatak}
% U preamble.tex je definisan \newtheorem{zadatak}{Zadatak}.
% Okruženje je automatski numerisano na isti način kao i \section itd.
% Komanda \dd daje pravilan znak diferencijala.
\begin{zadatak}
	\label{prvi-integral}
	Izračunati integral $I$ ako je $D$ oblast ograničena sa $y_1=2x^2, \ y_2 = 1+x^2$.
	\[
	I = \iint_D (x+2y) \ \dd x \dd y
	.\] 
\end{zadatak}

% Tikz može da generiše grafike na osnovu jednačina.
\begin{figure}[H]
	\centering
	\begin{tikzpicture}
		\begin{axis}[	ymin=-1,ymax=3,xmax=2,xmin=-2,%xticklabel=\empty,yticklabel=\empty,
			minor tick num=1,axis on top,axis lines = middle,
			xlabel=$x$,ylabel=$y$,label style={at={(ticklabel cs:1.1)}}]
			\addplot[blue, samples=50, domain=-1.5:1.5]{2*x^2};
			\addplot[red, samples=50, domain=-1.5:1.5]{1+x^2};
			\addplot[name path=A, blue, samples=50, domain=-1:1]{2*x^2};
			\addplot[name path=B,red, samples=50, domain=-1:1]{1+x^2};
			\node[label={180:{(-1,2)}},circle,fill,inner sep=1pt] at (axis cs:-1,2) {};
			\node[label={180:{(1,2)}},circle,fill,inner sep=1pt] at (axis cs:1,2) {};
			\addplot[blue!15] fill between[of=A and B];
		\end{axis}
	\end{tikzpicture}
	\caption{Slika uz zadatak \ref{prvi-integral}}
\end{figure}

Prvi korak je odrediti granice: $2x^2=1+x^2 \implies x = \pm 1$. Tačke presjeka su dakle $(1,2)$ i $(-1,2)$.

\[
x\Big|_{-1}^{1}, \qquad y \Big|_{2x^2}^{1+x^2}
.\] 

% Okruženje alignat vertikalno poravna znakove obilježene sa &.
% Bez zvjezdice daje numerisane linije.
% Numerisati samo jednu liniju se može sa \tag{1} na kraju linije.
\begin{alignat*}{1}
	I 	&= \int_{-1}^{1} \dd x \int_{2x^2}^{1+x^2}(x+2y) \ \dd y 
	= \int_{-1}^{1} \dd x \left( xy\Big|_{2x^2}^{1+x^2} + 2y^2 \Big|_{2x^2}^{1+x^2} \right) \\
	&= \int_{-1}^{1} \left\{ x(1+x^2-2x^2) + 2 \left[ (1+x^2)^2 - (2x^2)^2 \right] \right\} \dd x \\
	&= \ldots = \frac{32}{15}
.\end{alignat*}


\subsection{Tabela}
% Preuzeto sa:
% https://latex-tutorial.com/tutorials/tables/
\begin{table}[h]
	\begin{center}
		\begin{tabular}{l|S|r}
			\toprule % <-- Toprule here
			\textbf{Value 1} & \textbf{Value 2} & \textbf{Value 3}\\
			$\alpha$ & $\beta$ & $\gamma$ \\
			\midrule % <-- Midrule here
			1 & 1110.1 & a\\
			2 & 10.1 & b\\
			3 & 23.113231 & c\\
			\bottomrule % <-- Bottomrule here
		\end{tabular}
		\caption{Tabela s paketom booktabs.}
	\end{center}
\end{table}

\subsection{Kompleksnija tabela}
% Primjer tabele preuzet sa
% https://tex.stackexchange.com/questions/112343/beautiful-table-samples
\begin{table}[h]
	\centering
	\begin{tabular}{SSSSSSSS} \toprule
		{$m$} & {$\Re\{\underline{\mathfrak{X}}(m)\}$} & {$-\Im\{\underline{\mathfrak{X}}(m)\}$} & {$\mathfrak{X}(m)$} & {$\frac{\mathfrak{X}(m)}{23}$} & {$A_m$} & {$\varphi(m)\ /\ ^{\circ}$} & {$\varphi_m\ /\ ^{\circ}$} \\ \midrule
		1  & 16.128 & +8.872 & 16.128 & 1.402 & 1.373 & -146.6 & -137.6 \\
		2  & 3.442  & -2.509 & 3.442  & 0.299 & 0.343 & 133.2  & 152.4  \\
		3  & 1.826  & -0.363 & 1.826  & 0.159 & 0.119 & 168.5  & -161.1 \\
		4  & 0.993  & -0.429 & 0.993  & 0.086 & 0.08  & 25.6   & 90     \\ \midrule
		5  & 1.29   & +0.099 & 1.29   & 0.112 & 0.097 & -175.6 & -114.7 \\
		6  & 0.483  & -0.183 & 0.483  & 0.042 & 0.063 & 22.3   & 122.5  \\
		7  & 0.766  & -0.475 & 0.766  & 0.067 & 0.039 & 141.6  & -122   \\
		8  & 0.624  & +0.365 & 0.624  & 0.054 & 0.04  & -35.7  & 90     \\ \midrule
		9  & 0.641  & -0.466 & 0.641  & 0.056 & 0.045 & 133.3  & -106.3 \\
		10 & 0.45   & +0.421 & 0.45   & 0.039 & 0.034 & -69.4  & 110.9  \\
		11 & 0.598  & -0.597 & 0.598  & 0.052 & 0.025 & 92.3   & -109.3 \\ \bottomrule
	\end{tabular}
	\caption{Kompleksna tabela}
\end{table}

\subsection{Mjerne jedinice}

Paket \texttt{siunits} dozvoljava da se lako i pravilno pišu veličine koje imaju mjerne jedinice. Kako se često pogrešno radi: $g = 9.81 ms^{-2}$.
Kako je pravilno: $g = \SI{9.81}{\meter\per\second\squared}$.



