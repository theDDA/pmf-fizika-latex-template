% !TEX root = ./main.tex
% A4, 12pt font
\documentclass[12pt,a4paper,final]{article}

% Mogu se direktno unositi čćžđš.
\usepackage[utf8]{inputenc}

% Šire margine
%\usepackage[left=2.5cm,right=2.5cm,top=2.5cm,bottom=3.5cm]{geometry}
\usepackage{fullpage}

% Odkomentarisati za veći prored
%\renewcommand{\baselinestretch}{1.25}

% Uvlačenje paragrafa
\setlength{\parindent}{0pt}

% Lokalizacija
\usepackage[bosnian]{babel}

% Proširenje matematike
\usepackage{amsmath, amsfonts}

% Češće koristimo \varphi
\newcommand\vp{\varphi}

% Komanda \diffp{x}{y} za parcijalne izvode.
% Komanda \dd daje znak diferencijala koji je uspravan i malo odvojen
% od podintegralne funkcije. Npr. $\int x \dd{x}$
\usepackage{physics}
\usepackage[thinc]{esdiff}

% Vektorske operacije, komanda \vb{}
\usepackage{esvect}

% Slike
\usepackage{graphicx}

% Linkovi
\usepackage{hyperref}

% Bolja tipografija i kerning
\usepackage{microtype}

% Proširenje za tabele
\usepackage{booktabs}

% Proširenje za floats, potrebno za komandu [H] kojom se forsira tačno mjesto slike/tabele.
\usepackage{float}

% Lakše unošenje jedinica.
% Npr. $g = \SI{9.81}{\meter\per\second\squared}$
\usepackage{siunitx}

% Crtanje dijagrama u tikz
\usepackage{tikz}
\usepackage{pgfplots}
\pgfplotsset{compat=newest}
\usepgfplotslibrary{patchplots}
\usepgfplotslibrary{fillbetween}
% Crtanje kola u tikz
\usepackage[RPvoltages]{circuitikz}

% Podrška za prelome unutar captions, npr. \caption{Dug \\ tekst}
\usepackage{caption}
\captionsetup{justification=centering}

% Možemo okrenuti pojedinačne stranice na landscape, \begin{landscape}...\end{landscape}
% Korisno za široke tabele.
\usepackage{pdflscape}

% Dozvoljava prelamanje align okruženja na novu stranicu
\allowdisplaybreaks

% Slike i tabele se numerišu prema poglavlju
\numberwithin{figure}{section}
\numberwithin{table}{section}

% Dodavanje nekih funkcija kojih nema u latexu na način
% na koji se pišu na našim prostorima.
% Hiperbolne trigonometrijske:
\DeclareMathOperator{\tgh}{th}
% Vektorske operacije:
\DeclareMathOperator{\rrot}{rot}
\DeclareMathOperator{\ddiv}{div}
\DeclareMathOperator{\ggrad}{grad}

% Novo okruženje "Zadatak"
\newtheorem{zadatak}{Zadatak}

% Komande za unos podataka u naslovnu stranicu
% Ne diraj ako ne znaš šta radiš :)
\makeatletter
\newcommand\naslov[1]{\def\@naslov{#1}}
\newcommand\podnaslov[1]{\def\@podnaslov{#1}}
\newcommand\student[1]{\def\@student{#1}}
\newcommand\indeks[1]{\def\@indeks{#1}}
\newcommand\profesor[1]{\def\@profesor{#1}}
\newcommand\datum[1]{\def\@datum{#1}}