% !TEX root = ./main.tex
%%%%%%%%%%%%%%%%%%%%%%%%%%%%%%%%%%%%%%%%%%%%%%%%%%%%%%%%%%%%%%%%%%%%%%%%%%%%%%%%%%%%%%%%%%
% Template za studente PMF Fizika u Sarajevu. Prvobitno je namijenjen za seminarski rad,
% kao i za neku malu edukaciju oko latexa, ali se može uz male prepravke iskoristiti i
% za zadaću ili kao pomoć pri pravljenju seminarskog ili magistarskog rada.
%
% Ubačeno je dosta komentara i primjera da bi adaptacija bila što lakša, kao i neki
% korisni linkovi s primjerima.
%
% Inspirisano sličnim templateom sa ETF:
% https://www.overleaf.com/latex/templates/template-za-izvjestaje-elektrotehnicki-fakultet-sarajevo/kbvbhqnyntsg
%
% Autor: Damir Agačević, 2021.
% damir.agacevic@gmail.com
% https://github.com/theDDA/
% GNU GPLv3
%%%%%%%%%%%%%%%%%%%%%%%%%%%%%%%%%%%%%%%%%%%%%%%%%%%%%%%%%%%%%%%%%%%%%%%%%%%%%%%%%%%%%%%%%%

% A4, 12pt font
\documentclass[12pt,a4paper,final]{article}
% Mogu se direktno unositi čćžđš.
\usepackage[utf8]{inputenc}
% Šire margine
%\usepackage[left=2.5cm,right=2.5cm,top=2.5cm,bottom=3.5cm]{geometry}
\usepackage{fullpage}
% Lokalizacija
\usepackage[serbian]{babel}
% Proširenje matematike
\usepackage{amsmath, amsfonts}
% Komanda \dd daje znak diferencijala koji je uspravan i malo odvojen
% od podintegralne funkcije. Npr. $\int x \dd x$
\newcommand\dd{\ \mathrm{d} }
% Novo okruženje "Zadatak"
\newtheorem{zadatak}{Zadatak}
% Dozvoljava prelamanje align okruženja na novu stranicu
\allowdisplaybreaks
% Slike
\usepackage{graphicx}
% Linkovi
\usepackage{hyperref}
% Bolja tipografija i kerning
\usepackage{microtype}
% Proširenje za tabele
\usepackage{booktabs}
% Proširenje za floats, potrebno za komandu [H] kojom se forsira tačno mjesto slike/tabele.
\usepackage{float}
% Lakše unošenje jedinica.
% Npr. $g = \SI{9.81}{\meter\per\second\squared}$
\usepackage{siunitx}
% Crtanje dijagrama u tikz
\usepackage{tikz}
\usepackage{pgfplots}
\pgfplotsset{compat=newest}
\usepgfplotslibrary{patchplots}
\usepgfplotslibrary{fillbetween}
% Crtanje kola u tikz
\usepackage[RPvoltages]{circuitikz}

% Dodavanje nekih funkcija kojih nema u latexu na način
% na koji se pišu na našim prostorima.
% Hiperbolne trigonometrijske:
\DeclareMathOperator{\tgh}{th}
% Vektorske operacije:
\DeclareMathOperator{\rot}{rot}

% Komande za unos podataka u naslovnu stranicu
% Ne diraj ako ne znaš šta radiš :)
\makeatletter
\newcommand\naslov[1]{\def\@naslov{#1}}
\newcommand\podnaslov[1]{\def\@podnaslov{#1}}
\newcommand\student[1]{\def\@student{#1}}
\newcommand\indeks[1]{\def\@indeks{#1}}
\newcommand\profesor[1]{\def\@profesor{#1}}
\newcommand\datum[1]{\def\@datum{#1}}

\begin{document}
	
	%%%%%%%%%%%%%%%%%%%%%%%%%%%%%%%%%%%%%%%%%%%%%%%%%%%%%%%%%%%%%%%%%%%%%%%%%%%%%%%%%%%%%%
	%%%%%%%%%%%%%%%%%%%%%%%%%%%%%%% Popuniti ovaj dio %%%%%%%%%%%%%%%%%%%%%%%%%%%%%%%%%%%%
	%%%%%%%%%%%%%%%%%%%%%%%%%%%%%%%%%%%%%%%%%%%%%%%%%%%%%%%%%%%%%%%%%%%%%%%%%%%%%%%%%%%%%%
	% Ukoliko ne želite sva velika slova, ući u naslovna.tex i obrisati obilježeni dio.
	\naslov{naslov}
	\podnaslov{podnaslov}
	\student{Ime Studenta}
	\indeks{1234/F}
	% Kada se pišu skraćenice uvijek koristiti ~ umjesto razmaka iza tačke inače se
	% dobijaju neželjeni razmaci i prelomi jer latex smatra da je kraj rečenice.
	\profesor{prof.~dr.~Ime Profesora}
	% Po defaultu samo godina.
	\datum{\the\year.}
	%\datum{\today}
	%\datum{01.~april 2020.}
	%%%%%%%%%%%%%%%%%%%%%%%%%%%%%%%%%%%%%%%%%%%%%%%%%%%%%%%%%%%%%%%%%%%%%%%%%%%%%%%%%%%%%%
	%%%%%%%%%%%%%%%%%%%%%%%%%%%%%%%%%%%%%%%%%%%%%%%%%%%%%%%%%%%%%%%%%%%%%%%%%%%%%%%%%%%%%%
	
	% !TEX root = ./main.tex
%%%%%%%%%%%%%%%%%%%%%%%%%%%%%%%%%%%%%%%%%%%%%%%%%%%%%%%%%%%%%%%%%%%%%%%%%%%%%%%%%%%%%%%%%%
% Template za studente PMF Fizika u Sarajevu. Prvobitno je namijenjen za seminarski rad,
% kao i za neku malu edukaciju oko latexa, ali se može uz male prepravke iskoristiti i
% za zadaću ili kao pomoć pri pravljenju seminarskog ili magistarskog rada.
%
% Ubačeno je dosta komentara i primjera da bi adaptacija bila što lakša, kao i neki
% korisni linkovi s primjerima.
%
% Inspirisano sličnim templateom sa ETF:
% https://www.overleaf.com/latex/templates/template-za-izvjestaje-elektrotehnicki-fakultet-sarajevo/kbvbhqnyntsg
%
% Autor: Damir Agačević, 2021.
% damir.agacevic@gmail.com
% https://github.com/theDDA/
% GNU GPLv3
%%%%%%%%%%%%%%%%%%%%%%%%%%%%%%%%%%%%%%%%%%%%%%%%%%%%%%%%%%%%%%%%%%%%%%%%%%%%%%%%%%%%%%%%%%

% Preporučujem ne dirati ako ne znaš šta radiš :)

\begin{titlepage}
	\begin{figure}[t]
		\centering
		\begin{minipage}{.6\textwidth}
			\centering
			\textsc{\large Univerzitet u Sarajevu}\\
			\textsc{\large Prirodno-matematički fakultet}\\
			\textsc{\large Odsjek za fiziku}
		\end{minipage}
	\end{figure}
	\vspace*{\fill}
	\begin{center}
		% Obrisati \MakeUppercase u ova dva reda ukoliko nisu potrebna velika slova.
		\textbf{\Large \MakeUppercase\@naslov}\\
		\vspace{2mm}
		\large \MakeUppercase\@podnaslov
	\end{center}
	\vspace*{\fill}
	\begin{figure}[b]
		\centering
		\begin{minipage}{.325\textwidth}
			\textbf{Student:}\\
			\@student, \@indeks \\
		\end{minipage}
		\begin{minipage}{.30\textwidth}
			\begin{center}
				% Ako neko zna bolji način da se ovo uradi, a sigurno ima,
				% neka mi se javi na mail pa izmijenim.
				\vspace{25mm}
				Sarajevo, \@datum
			\end{center}
		\end{minipage}
		\begin{minipage}{.325\textwidth}
			\textbf{Profesor: }\\
			\@profesor \\
		\end{minipage}
	\end{figure}
\end{titlepage}
	% Nemojte zaboraviti: da bi se sadržaj, spisak, reference, fusnote itd. pravilno
	% prikazali, morate dvaput kompajlirati. Neki softverski paketi
	% to urade sami, ali ako vam nešto nije uredu probajte prvo ručno dva puta
	% pritisnuti compile/build/run (kako se već zove u vašem softveru).
	% Ukoliko radite online u Overleafu (ili, naprednije, ako koristite latexmk),
	% ovo bi trebalo biti automatski.
	\tableofcontents
	\newpage
	\listoffigures
	% Ubaciti \newpage ako hoćete da su spiskovi svaki na svojoj stranici.
	\listoftables
	
	% Dobra je praksa odvajati veće odjeljke svaki u svoj fajl, mada za seminarske radove
	% nema tolike potrebe za time.
	% Napomena: Ovim odvajanjem svaki section počinje na novoj stranici.
	%%%%%%%%%%%%%%%%%%%%%%%%%%%%%%%%%%%%%%%%%%%%%%%%%%%%%%%%%%%%%%%%%%%%%%%%%%%%%%%%%%%%%%
%%%%%%%%%%%%%%%%%%%%%%%%%%%%%%% Obrisati ovaj dio %%%%%%%%%%%%%%%%%%%%%%%%%%%%%%%%%%%%
%%%%%%%%%%%%%%%%%%%%%%%%%%%%%%%%%%%%%%%%%%%%%%%%%%%%%%%%%%%%%%%%%%%%%%%%%%%%%%%%%%%%%%

\section{Naslov}

Lorem ipsum dolor sit amet, consectetur adipiscing elit, sed do eiusmod tempor incididunt ut labore et dolore magna aliqua. Ut enim ad minim veniam, quis nostrud exercitation ullamco laboris nisi ut aliquip ex ea commodo consequat. Duis aute irure dolor in reprehenderit in voluptate velit esse cillum dolore eu fugiat nulla pariatur. Excepteur sint occaecat cupidatat non proident, sunt in culpa qui officia deserunt mollit anim id est laborum.

\subsection{Podnaslov}

Lorem ipsum dolor sit amet, consectetur adipiscing elit, sed do eiusmod tempor incididunt ut labore et dolore magna aliqua. Nulla porttitor massa id neque aliquam vestibulum morbi blandit. Nunc scelerisque viverra mauris in. Facilisis volutpat est velit egestas dui id ornare. Sollicitudin nibh sit amet commodo nulla facilisi nullam vehicula. Elementum sagittis vitae et leo duis ut diam quam. Lorem sed risus ultricies tristique nulla aliquet enim tortor at. Volutpat odio facilisis mauris sit. At tellus at urna condimentum mattis pellentesque id nibh. A cras semper auctor neque vitae tempus quam pellentesque. Venenatis a condimentum vitae sapien pellentesque habitant. Purus gravida quis blandit turpis cursus in hac habitasse platea. Aenean et tortor at risus viverra. Quisque sagittis purus sit amet volutpat consequat mauris. Sagittis orci a scelerisque purus semper eget duis at tellus. Iaculis urna id volutpat lacus. Urna duis convallis convallis tellus id interdum. Praesent tristique magna sit amet. Netus et malesuada fames ac turpis egestas sed tempus. \\

In fermentum posuere urna nec. Massa sapien faucibus et molestie ac feugiat sed lectus. Sit amet volutpat consequat mauris. Ac tincidunt vitae semper quis. Platea dictumst quisque sagittis purus. Ultricies leo integer malesuada nunc vel risus commodo viverra. Commodo viverra maecenas accumsan lacus vel facilisis volutpat est. Montes nascetur ridiculus mus mauris vitae ultricies leo integer malesuada. Egestas integer eget aliquet nibh praesent. Quis ipsum suspendisse ultrices gravida. Sociis natoque penatibus et magnis dis parturient montes. Pellentesque diam volutpat commodo sed egestas egestas fringilla phasellus. In mollis nunc sed id semper risus. \\

\subsubsection{Podpodnaslov}
Hac habitasse platea dictumst quisque sagittis. Et pharetra pharetra massa massa ultricies mi quis hendrerit dolor. Lectus magna fringilla urna porttitor rhoncus dolor purus non enim. Ac tincidunt vitae semper quis lectus. Sit amet risus nullam eget felis eget nunc lobortis mattis. Nulla pellentesque dignissim enim sit amet venenatis urna cursus eget. Turpis massa sed elementum tempus egestas sed sed risus pretium. Risus feugiat in ante metus dictum at tempor commodo. Posuere morbi leo urna molestie at. Volutpat diam ut venenatis tellus in metus vulputate eu. Pulvinar etiam non quam lacus suspendisse. Molestie a iaculis at erat pellentesque adipiscing commodo. Sed euismod nisi porta lorem.

	%%%%%%%%%%%%%%%%%%%%%%%%%%%%%%%%%%%%%%%%%%%%%%%%%%%%%%%%%%%%%%%%%%%%%%%%%%%%%%%%%%%%%%
%%%%%%%%%%%%%%%%%%%%%%%%%%%%%%% Obrisati ovaj dio %%%%%%%%%%%%%%%%%%%%%%%%%%%%%%%%%%%%
%%%%%%%%%%%%%%%%%%%%%%%%%%%%%%%%%%%%%%%%%%%%%%%%%%%%%%%%%%%%%%%%%%%%%%%%%%%%%%%%%%%%%%

\section{Primjeri}

\subsection{Slika}
\begin{figure}[h]
	\begin{center}
		\includegraphics[width=0.3\textwidth]{pmf.png}
		\caption{Logo PMF koji nije Word 2003 clip art}
	\end{center}
\end{figure}

\subsection{Zadatak}
% U preamble.tex je definisan \newtheorem{zadatak}{Zadatak}.
% Okruženje je automatski numerisano na isti način kao i \section itd.
% Komanda \dd daje pravilan znak diferencijala.
\begin{zadatak}
	\label{prvi-integral}
	Izračunati integral $I$ ako je $D$ oblast ograničena sa $y_1=2x^2, \ y_2 = 1+x^2$.
	\[
	I = \iint_D (x+2y) \ \dd x \dd y
	.\] 
\end{zadatak}

% Tikz može da generiše grafike na osnovu jednačina.
\begin{figure}[H]
	\centering
	\begin{tikzpicture}
		\begin{axis}[	ymin=-1,ymax=3,xmax=2,xmin=-2,%xticklabel=\empty,yticklabel=\empty,
			minor tick num=1,axis on top,axis lines = middle,
			xlabel=$x$,ylabel=$y$,label style={at={(ticklabel cs:1.1)}}]
			\addplot[blue, samples=50, domain=-1.5:1.5]{2*x^2};
			\addplot[red, samples=50, domain=-1.5:1.5]{1+x^2};
			\addplot[name path=A, blue, samples=50, domain=-1:1]{2*x^2};
			\addplot[name path=B,red, samples=50, domain=-1:1]{1+x^2};
			\node[label={180:{(-1,2)}},circle,fill,inner sep=1pt] at (axis cs:-1,2) {};
			\node[label={180:{(1,2)}},circle,fill,inner sep=1pt] at (axis cs:1,2) {};
			\addplot[blue!15] fill between[of=A and B];
		\end{axis}
	\end{tikzpicture}
	\caption{Slika uz zadatak \ref{prvi-integral}}
\end{figure}

Prvi korak je odrediti granice: $2x^2=1+x^2 \implies x = \pm 1$. Tačke presjeka su dakle $(1,2)$ i $(-1,2)$.

\[
x\Big|_{-1}^{1}, \qquad y \Big|_{2x^2}^{1+x^2}
.\] 

% Okruženje alignat vertikalno poravna znakove obilježene sa &.
% Bez zvjezdice daje numerisane linije.
% Numerisati samo jednu liniju se može sa \tag{1} na kraju linije.
\begin{alignat*}{1}
	I 	&= \int_{-1}^{1} \dd x \int_{2x^2}^{1+x^2}(x+2y) \ \dd y 
	= \int_{-1}^{1} \dd x \left( xy\Big|_{2x^2}^{1+x^2} + 2y^2 \Big|_{2x^2}^{1+x^2} \right) \\
	&= \int_{-1}^{1} \left\{ x(1+x^2-2x^2) + 2 \left[ (1+x^2)^2 - (2x^2)^2 \right] \right\} \dd x \\
	&= \ldots = \frac{32}{15}
.\end{alignat*}


\subsection{Tabela}
% Preuzeto sa:
% https://latex-tutorial.com/tutorials/tables/
\begin{table}[h]
	\begin{center}
		\begin{tabular}{l|S|r}
			\toprule % <-- Toprule here
			\textbf{Value 1} & \textbf{Value 2} & \textbf{Value 3}\\
			$\alpha$ & $\beta$ & $\gamma$ \\
			\midrule % <-- Midrule here
			1 & 1110.1 & a\\
			2 & 10.1 & b\\
			3 & 23.113231 & c\\
			\bottomrule % <-- Bottomrule here
		\end{tabular}
		\caption{Tabela s paketom booktabs.}
	\end{center}
\end{table}

\subsection{Kompleksnija tabela}
% Primjer tabele preuzet sa
% https://tex.stackexchange.com/questions/112343/beautiful-table-samples
\begin{table}[h]
	\centering
	\begin{tabular}{SSSSSSSS} \toprule
		{$m$} & {$\Re\{\underline{\mathfrak{X}}(m)\}$} & {$-\Im\{\underline{\mathfrak{X}}(m)\}$} & {$\mathfrak{X}(m)$} & {$\frac{\mathfrak{X}(m)}{23}$} & {$A_m$} & {$\varphi(m)\ /\ ^{\circ}$} & {$\varphi_m\ /\ ^{\circ}$} \\ \midrule
		1  & 16.128 & +8.872 & 16.128 & 1.402 & 1.373 & -146.6 & -137.6 \\
		2  & 3.442  & -2.509 & 3.442  & 0.299 & 0.343 & 133.2  & 152.4  \\
		3  & 1.826  & -0.363 & 1.826  & 0.159 & 0.119 & 168.5  & -161.1 \\
		4  & 0.993  & -0.429 & 0.993  & 0.086 & 0.08  & 25.6   & 90     \\ \midrule
		5  & 1.29   & +0.099 & 1.29   & 0.112 & 0.097 & -175.6 & -114.7 \\
		6  & 0.483  & -0.183 & 0.483  & 0.042 & 0.063 & 22.3   & 122.5  \\
		7  & 0.766  & -0.475 & 0.766  & 0.067 & 0.039 & 141.6  & -122   \\
		8  & 0.624  & +0.365 & 0.624  & 0.054 & 0.04  & -35.7  & 90     \\ \midrule
		9  & 0.641  & -0.466 & 0.641  & 0.056 & 0.045 & 133.3  & -106.3 \\
		10 & 0.45   & +0.421 & 0.45   & 0.039 & 0.034 & -69.4  & 110.9  \\
		11 & 0.598  & -0.597 & 0.598  & 0.052 & 0.025 & 92.3   & -109.3 \\ \bottomrule
	\end{tabular}
	\caption{Kompleksna tabela}
\end{table}

\subsection{Mjerne jedinice}

Paket \texttt{siunits} dozvoljava da se lako i pravilno pišu veličine koje imaju mjerne jedinice. Kako se često pogrešno radi: $g = 9.81 ms^{-2}$.
Kako je pravilno: $g = \SI{9.81}{\meter\per\second\squared}$.





	\newpage
	% Izbacuje sve izvore koji su u main.bib.
	% Ukloniti da izbacuje samo izvore koji su eksplicitno citirani s \cite{}.
	\nocite{*}
	% Zamijeniti "Izvori" s npr. "Bibliografija" itd.
	\printbibliography[title={Izvori},heading=bibintoc]
	
\end{document}